%!TEX root = data_poisoning.tex
% Добавьте ссылку на файлы с текстом работы
% Можно использовать команды:
%   \input или \include
% Пример:
%    \input{mainfiles/1-section} или \include{mainfiles/2-section}
% Команда \input позволяет включить текст файла без дополнительной обработки
% Команда \include при включении файла добавляет до него и после него команду
% перехода на новую страницу. Кроме того, она позволяет компилировать каждый файл
% в отдельности, что ускоряет сборку проекта.
% ВАЖНО: команда \include не поддерживает включение файлов, в которых уже содержится команда \include,
% т.е. не возможен рекурсивный вызов \include
\newcommand*{\Source}{
    %!TEX root = ../data_poisoning.tex
\phantomsection
\section*{Введение}
\addcontentsline{toc}{section}{Введение}
Системы моделей машинного обучения быстро увеличиваются в размерах, приобретают новые возможности и все чаще используются в условиях повышенной ответственности. Как и в случае с другими технологиями, безопасность для моделей машинного обучения должна быть одним из основных исследовательских приоритетов.~\cite{hendrycks_unsolved_2021}

Зачастую сложные обученные модели плохо поддаются интерпретации, и, в большой степени, представляют собой чёрный ящик.
Это свойство делает их уязвимыми к различным отравляющим атакам, поскольку проверка корректности работы модели на всём пространстве признаков зачастую является практически невыполнимой задачей.\cite{chen_detecting_2018}

В данной работе в основном рассмотрены методы отравления обучающих данных на примере моделей для классификации изображений. Однако идеи данных методов также могут быть применены для других модальностей данных и задач машинного обучения.


\subsection*{Обозначения}
\begin{itemize}
    \item ${X_b}$ – множество чистых изображений;
    \item ${X_p}$ – множество отравленных изображений;
    \item ${x_{b} \in X_b}$ – чистое изображение;
    \item ${x_{p} \in X_p}$ – отравленное изображение;
    \item $f(x)$ – функция извлечения признаков.
\end{itemize}

    %!TEX root = ../data_poisoning.tex
\phantomsection
\section*{Постановка задачи}
\addcontentsline{toc}{section}{Постановка задачи}

\subsection*{Цели}
\begin{itemize}
    \item Провести обзор и сравнение методов атак отравлением данных;
    \item Провести обзор и сравнение методов защиты от атак отравлением данных;
    \item Провести практические эксперименты с набором данных MNIST используя.
\end{itemize}

\subsection*{Задачи}
\begin{itemize}
    \item Реализовать атаку отравлением набора данных MNIST посредством методов:
    \begin{itemize}
        \item Clean-label backdoor;
        \item Feature Collision;
        \item Substitution of labels;
    \end{itemize}
    \item Провести анализ набора данных при помощи методов кластеризации с целью выделения отравленных данных;
    \item Провести анализ сравнительный эффективности атак.
\end{itemize}

    %!TEX root = ../data_poisoning.tex
\section{Атаки отравлением данных}
\subsection*{Цель}
Изменить набор обучающий данных $D$ таким образом, чтобы целевое изображение $x_t \in D$, имеющее класс $C$ неверно классифицировалось в целевой класс $C'$.
При этом нужно оказать как можно меньшее влияние на классификацию остальных изображений.

\subsection*{Дополнительные обозначения}
\begin{itemize}
    \item ${x_{b}}$ – чистое изображение;
    \item ${x_{p} \in X_p}$ – отравленное изображение;
    \item $f(x)$ – функция извлечения признаков;
\end{itemize}

\subsection*{Классические методы отравления данных}

\subsubsection*{Модель угроз}
Атакующий контролирует обучающий данных $D$ и может произвольно менять в нём данные в пределах $J$ примеров.

\subsubsection*{Описание атак}
\textbf{Простая замена метки класса у целевого изображения}
Очень малоэффективная атака при больших объёмах доступных для обучения данных. Зачастую модели обладают хорошей обобщающей способностью и с большой вероятностью не выучат один неверно размеченный пример.

\textbf{Feature Collision}
Вместе с простой заменой метки класса у целевого изображения, мы пытаемся осуществить минимальное изменение изображений из целевого класса $C'$ так чтобы признаки отравленных данных были максимально близки к признакам целевого изображения.

$$x^{j}_{p} = \argmin_x ||f(x)-f(x_t)||_{2}^{2}$$

$$\text{при условии, что~} ||x_{p}^{j} - x_{b}^{j}||_{\infty} \leq \epsilon$$


\textbf{Convex Polytope}
Вместе с простой заменой метки класса у целевого изображения, мы пытаемся сделать так, чтобы признаки целевого изображения представлялись линейной комбинацией отравленных данных из класса $C'$

$$X_p = \argmin_{c_j, x^j} ||f(x_t) - \sum_{j=1}^{J} c_j f(x^j){||}_{2}^{2}$$
$$\text{при условии~} \sum_{j=1}^{J} c_j = 1$$
$$\text{и~} c_j \geq 0~\forall j$$
$$\text{и~} ||x_{p}^{j} - x_{b}^{j}||_{\infty} \leq \epsilon$$


\textbf{Witches' Brew via gradient matching}
При данной атаке нам дополнительно известна архитектура обучаемой модели $F(x, \theta)$ и её функция потерь $L(y_{pred}, y_{true})$.
Вместе с простой заменой метки класса у целевого изображения, мы отравляем данные из класса $C'$ так чтобы вектора градиентов функции потерь для них были сонаправлены с вектором градиента функции потерь для целевого изображения.

$$\nabla_{\theta} L(F(x_t, \theta), C') \uparrow\uparrow \nabla_{\theta} L(F(x_{p}^{j}, \theta), C')$$
$$\text{где~} x_{p}^{j} = x_{b}^{j} + \delta_j \text{~и~} ||\delta_j||_{\infty} \leq \epsilon$$

\subsection*{Методы атак без доступа к полному набору обучающих данных}


\subsubsection*{Модель угроз}
В атаках без возможности чтения обучающих данных (также называются атаки методом чёрного ящика) ожидается, что атакующий не имеет доступа к обучающему набору данных, однако знает из какого они распределения и имеет возможность выбирать данные из этого распределения неограниченно, либо по определённым правилам. Также атакующий знает структуру обучаемой модели, её функцию потерь и имеет возможность внедрять отравленные данные в обучающий набор данных для конечной модели.

\textbf{Subpopulation Attack}
Идея заключается в следующем: мы выбираем большое количество данных из известного нам распределения исходных данных. Затем проводим кластеризацию.

Кластеризацию можно производить:
\begin{itemize}
    \item Вручную по высокоуровнему признаку: например наличия на картинке какого-либо элемента
    \item Автоматически путём предварительной фильтрации и кластеризации либо оригинальных изображений, либо извлечённых признаков
\end{itemize}

Кластеризацию следует производить таким образом, чтобы кластер с целевым изображением имел размер не более чем $J$. Затем мы меняем метку на $C'$ для всего кластера и отправляем получившиеся данные на обучение модели.

% TODO: описать оптимизацию атаки

\textbf{Online Learning Attack}
В атаке предполагается что модель обучается в режиме реального времени и атакующему известно текущие веса модели. В каждый момент времени атакующий получает обучающий пример и у него есть два варианта действия:
\begin{itemize}
    \item Передать пример модели
    \item Модифицировать пример и передать модели
\end{itemize}

При этом атакующий не может менять принятого ранее решения.

Задачу можно переформулировать в виде марковского процесса принятия решений, введя функцию штрафов $g(s_T, x_T)$, где $s_T$ – текущее состояние, описываемое положением в марковском процессе и весами модели, $x_T$ – пример, доступный в момент времени $T$.

\noindentТогда ожидаемый штраф для атакующего будет составлять:
$$V_{\mathcal{M}}^{\phi}(\mathbf{s}):=\left.\mathbb{E}_{\mathcal{M}} \sum_{T=0}^{\infty} \gamma^{T} g\left(\mathbf{s}_{T}, \phi\left(\mathbf{s}_{T}\right)\right)\right|_{\mathbf{s0}=\mathbf{s}}$$
Где $\gamma$ – дисконтирующий множитель, введённый для решения проблемы с неограниченностью процесса по времени.


\noindentА оптимальная политика действий:
$$
\phi* = \argmin_{\phi} \mathbb{E}_{\mathbf{S} \sim \mu_{0}} V_{\mathcal{M}}^{\phi}(\mathbf{s})
$$

    %!TEX root = ../data_poisoning.tex
\section{Практические результаты}

Тут будет таблица с результатами экспериментов
    %!TEX root = ../data_poisoning.tex
\phantomsection
\section*{Вывод}
\addcontentsline{toc}{section}{Вывод}
Обзор современных методов атак отравлением обучающих данных на нейросетевые модели показывает, что на данный момент такие атаки активно развиваются и представляют собой реальную угрозу ввиду своей эффективности в определенных конфигурациях. Например, при определённых условиях, при отравлении 1\% обучающего набора данных можно достигнуть поставленной цели на более чем 90\% моделей обученных на этих данных.~[3]


Тут выводы о практических результатах.
}


% Информация о годе выполнения работы
\def\Year{%
    % 2006%
    \the\year%     % Текущий год
}

% Укажите тип работы
% Например:
%     Выпускная квалификационная работа,
%     Магистерская диссертация,
%     Курсовая работа, реферат и т.п.
\def\WorkType{%
    % Выпускная квалификационная работа%
    % Магистерская диссертация%
    Курсовая работа%
    % Реферат%
    % Дипломная работа%
}

% Название работы
%%%%%%%%%%% ВНИМАНИЕ! %%%%%%%%%%%%%%%%
% В МГУ ОНО ДОЛЖНО В ТОЧНОСТИ
% СООТВЕТСТВОВАТЬ ВЫПИСКЕ ИЗ ПРИКАЗА
% УТОЧНИТЕ НАЗВАНИЕ В УЧЕБНОЙ ЧАСТИ
\def\Title{%
    Исследование атак отравлением данных и методов защиты от них%
}


% Имя автора работы
\def\Author{%
    Лозинский Иван Павлович%
}

% Информация о научном руководителе
%% Фамилия Имя Отчество%
\def\SciAdvisor{%
    Саада Даниель Фирасович%
}
%% В формате: И.~О.~Фамилия%
\def\SciAdvisorShort{%
    Д.~Ф.~Саада%
}
%% должность научного руководителя
\def\Position{%
    % профессор%
    % доцент%
    % старший преподаватель%
    % преподаватель%
    % ассистент%
    % ведущий научный сотрудник%
    % старший научный сотрудник%
    % научный сотрудник%
    % младший научный сотрудник%
}
%% учёная степень научного руководителя
\def\AcademicDegree{%
    % д.ф.-м.н.%
    % д.т.н.%
    % к.ф.-м.н.%
    % к.т.н.%
    % без степени%
}

% Информация об организации, в которой выполнена работа
%% Город
\def\Place{%
    Москва%
}
%% Университет
\def\Univer{%
    Московский государственный университет имени М.~В.~Ломоносова%
}
%% Факультет
\def\Faculty{%
    Факультет вычислительной математики и кибернетики%
}
%% Кафедра
\def\Department{%
    Кафедра информационной безопасности%
}

%%%% Переключите статус документа для отладки
%%%% В режиме draft документ собирается очень быстро
%%%% и выводится полезная информация о том
%%%% какие строки вылезают за границы документа, что удобно для борьбы с ними
\def\Status{%
    % draft%
    final%
}

%%%% Включает и выключает подпись <<С текстом работы ознакомлен>>
\def\EnableSign{%
    % true%
}
