%!TEX root = data_poisoning.tex
% Добавьте ссылку на файлы с текстом работы
% Можно использовать команды:
%   \input или \include
% Пример:
%    \input{mainfiles/1-section} или \include{mainfiles/2-section}
% Команда \input позволяет включить текст файла без дополнительной обработки
% Команда \include при включении файла добавляет до него и после него команду
% перехода на новую страницу. Кроме того, она позволяет компилировать каждый файл
% в отдельности, что ускоряет сборку проекта.
% ВАЖНО: команда \include не поддерживает включение файлов, в которых уже содержится команда \include,
% т.е. не возможен рекурсивный вызов \include
\newcommand*{\Source}{
    %!TEX root = ../data_poisoning.tex
\phantomsection
\section*{Цели и задачи}
\addcontentsline{toc}{section}{Цели и задачи}

\subsection*{Цели}
\begin{itemize}
    \item Провести обзор и сравнение методов атак отравлением данных
    \item Провести обзор и сравнение методов защиты от атак отравлением данных
    \item Провести практические эксперименты с набором данных MNIST используя
\end{itemize}

\subsection*{Постановка задачи}
\begin{itemize}
    \item Реализовать атаку отравлением набора данных MNIST посредством методов:
    \begin{itemize}
        \item Clean-label backdoor
        \item Feature Collision
        \item Substitution of labels
    \end{itemize}
    \item Провести анализ набора данных при помощи методов кластеризации с целью выделения отравленных данных
    \item Провести анализ сравнительный эффективности атак
\end{itemize}

    %!TEX root = ../data_poisoning.tex
\phantomsection
\section*{Введение} 
\addcontentsline{toc}{section}{Введение}
Криптосистема Мак-Элиса "--- одна из старейших криптосистем с открытым ключом. Она была предложена в 1978
Р.~Дж.~Мак-Элисом~\cite{MCEliece}. Данная криптосистема
основывается на $\mathbf{\mathbb {NP}}$-трудной проблеме в теории
кодирования. Основная идея её построения  состоит в маскировке
некоторого кода, имеющего эффективные алгоритмы декодирования, под
код, не обладающий видимой алгебраической и комбинаторной
структурой, такие коды принято называть кодами общего положения.
Эта криптосистема обладает одним важным преимуществом "--- высокой
скоростью зашифрования и расшифрования. Однако, у неё имеется
серьёзный недостаток "--- относительно низкая скорость передачи
($R$). Обычно у кодовых криптосистем $R<1$, тогда как у
криптосистемы RSA скорость в точности равна $1$.

В этой работе рассматривается обобщение криптосистемы
Мак-Элиса, предложенное в 1994 коду В.М.
Сидельниковым~\cite{Sidelnikov1}. В этой работе модификация,
предложенная В.~М.~Сидельниковым, называется криптосистемой\\
Мак-Элиса--Сидельникова. Криптосистема Мак-Элиса--Сидельникова
строится на основе $u$-кратного использования кодов Рида--Маллера
$RM(r,m)$. Она имеет высокую криптографическую стойкость, скорость
передачи близкую к $1$ и сравнительно невысокую сложность
шифрования секретных сообщений и расшифрования криптограмм этих
сообщений.

В работе исследуются вопросы, связанные с пространством
эквивалентных секретных ключей, то есть секретных ключей,
порождающих одинаковые открытые ключи, новой криптосистемы. Опишем
краткое содержание разделов работы.

В \S~1 даётся определение криптосистемы Мак-Элиса, описываются её
секретный и открытый ключи. Приводятся алгоритмы зашифрования и
расшифрования.

В \S~2 изучается ключевое пространство криптосистемы Мак-Элиса.
Устанавливается связь классов эквивалентностей секретных ключей с
группой автоморфизмов линейного кода, лежащего в основе этой
криптосистемы.

В \S~3 описывается криптосистема Мак-Элиса--Сидельникова:
секретный и открытый ключи, алгоритмы зашифрования и
расшифрования.

\S~4 посвящён ключевому пространству новой криптосистемы. В нём
вводятся множества, необходимые для описания классов
эквивалентности секретных ключей. Получаются нижние и верхние
оценки на мощности  введённых множеств и на число открытых ключей
криптосистемы Мак-Элиса--Сидельникова.

В \S~5 изучается криптосистема Мак-Элиса--Сидельникова в случае
двух блоков ($u=2$).

В настоящей работе получаются нижние оценки на мощность множества
открытых ключей криптосистемы
Мак-Элиса--Сидельникова(теорема~\ref{t3}) при использовании
произвольного числа блоков $u$. Для кодов Рида--Маллера с
$u$-кратным повторением строится множество, которое, в некотором
смысле, является аналогом группы автоморфизмов обычного кода
Рида--Маллера, и устанавливается связь этого множества с классами
эквивалентности секретных ключей.

Для случая двух блоков ($u=2$) полностью описывается указанное
множество при использовании кодов Рида--Маллера $RM(r,m)$
$(r\leqslant 2,r<m)$ и матриц определённого вида
(теоремы~\ref{theorem1},~\ref{theorem2}). Тем самым при $u=2,
r\geqslant 2, r<m$ описываются все классы эквивалентности
секретных ключей с представителями особого вида и вычисляются их
мощности. Для некоторых классов эквивалентности секретных ключей
приводятся нижние оценки на их мощность(теоремы~\ref{theorem1}
и~\ref{theorem2}).

    %!TEX root = ../data_poisoning.tex
\section{Атаки отравлением данных}
\subsection*{Цель}
Изменить набор обучающий данных $D$ таким образом, чтобы целевое изображение $x_t \in D$, имеющее класс $C$ неверно классифицировалось в целевой класс $C'$.
При этом нужно оказать как можно меньшее влияние на классификацию остальных изображений.

\subsection*{Дополнительные обозначения}
\begin{itemize}
    \item ${x_{b}}$ – чистое изображение;
    \item ${x_{p} \in X_p}$ – отравленное изображение;
    \item $f(x)$ – функция извлечения признаков;
\end{itemize}

\subsection*{Классические методы отравления данных}

\subsubsection*{Модель угроз}
Атакующий контролирует обучающий данных $D$ и может произвольно менять в нём данные в пределах $J$ примеров.

\subsubsection*{Описание атак}
\textbf{Простая замена метки класса у целевого изображения}
Очень малоэффективная атака при больших объёмах доступных для обучения данных. Зачастую модели обладают хорошей обобщающей способностью и с большой вероятностью не выучат один неверно размеченный пример.

\textbf{Feature Collision}
Вместе с простой заменой метки класса у целевого изображения, мы пытаемся осуществить минимальное изменение изображений из целевого класса $C'$ так чтобы признаки отравленных данных были максимально близки к признакам целевого изображения.

$$x^{j}_{p} = \argmin_x ||f(x)-f(x_t)||_{2}^{2}$$

$$\text{при условии, что~} ||x_{p}^{j} - x_{b}^{j}||_{\infty} \leq \epsilon$$


\textbf{Convex Polytope}
Вместе с простой заменой метки класса у целевого изображения, мы пытаемся сделать так, чтобы признаки целевого изображения представлялись линейной комбинацией отравленных данных из класса $C'$

$$X_p = \argmin_{c_j, x^j} ||f(x_t) - \sum_{j=1}^{J} c_j f(x^j){||}_{2}^{2}$$
$$\text{при условии~} \sum_{j=1}^{J} c_j = 1$$
$$\text{и~} c_j \geq 0~\forall j$$
$$\text{и~} ||x_{p}^{j} - x_{b}^{j}||_{\infty} \leq \epsilon$$


\textbf{Witches' Brew via gradient matching}
При данной атаке нам дополнительно известна архитектура обучаемой модели $F(x, \theta)$ и её функция потерь $L(y_{pred}, y_{true})$.
Вместе с простой заменой метки класса у целевого изображения, мы отравляем данные из класса $C'$ так чтобы вектора градиентов функции потерь для них были сонаправлены с вектором градиента функции потерь для целевого изображения.

$$\nabla_{\theta} L(F(x_t, \theta), C') \uparrow\uparrow \nabla_{\theta} L(F(x_{p}^{j}, \theta), C')$$
$$\text{где~} x_{p}^{j} = x_{b}^{j} + \delta_j \text{~и~} ||\delta_j||_{\infty} \leq \epsilon$$

\subsection*{Методы атак без доступа к полному набору обучающих данных}


\subsubsection*{Модель угроз}
В атаках без возможности чтения обучающих данных (также называются атаки методом чёрного ящика) ожидается, что атакующий не имеет доступа к обучающему набору данных, однако знает из какого они распределения и имеет возможность выбирать данные из этого распределения неограниченно, либо по определённым правилам. Также атакующий знает структуру обучаемой модели, её функцию потерь и имеет возможность внедрять отравленные данные в обучающий набор данных для конечной модели.

\textbf{Subpopulation Attack}
Идея заключается в следующем: мы выбираем большое количество данных из известного нам распределения исходных данных. Затем проводим кластеризацию.

Кластеризацию можно производить:
\begin{itemize}
    \item Вручную по высокоуровнему признаку: например наличия на картинке какого-либо элемента
    \item Автоматически путём предварительной фильтрации и кластеризации либо оригинальных изображений, либо извлечённых признаков
\end{itemize}

Кластеризацию следует производить таким образом, чтобы кластер с целевым изображением имел размер не более чем $J$. Затем мы меняем метку на $C'$ для всего кластера и отправляем получившиеся данные на обучение модели.

% TODO: описать оптимизацию атаки

\textbf{Online Learning Attack}
В атаке предполагается что модель обучается в режиме реального времени и атакующему известно текущие веса модели. В каждый момент времени атакующий получает обучающий пример и у него есть два варианта действия:
\begin{itemize}
    \item Передать пример модели
    \item Модифицировать пример и передать модели
\end{itemize}

При этом атакующий не может менять принятого ранее решения.

Задачу можно переформулировать в виде марковского процесса принятия решений, введя функцию штрафов $g(s_T, x_T)$, где $s_T$ – текущее состояние, описываемое положением в марковском процессе и весами модели, $x_T$ – пример, доступный в момент времени $T$.

\noindentТогда ожидаемый штраф для атакующего будет составлять:
$$V_{\mathcal{M}}^{\phi}(\mathbf{s}):=\left.\mathbb{E}_{\mathcal{M}} \sum_{T=0}^{\infty} \gamma^{T} g\left(\mathbf{s}_{T}, \phi\left(\mathbf{s}_{T}\right)\right)\right|_{\mathbf{s0}=\mathbf{s}}$$
Где $\gamma$ – дисконтирующий множитель, введённый для решения проблемы с неограниченностью процесса по времени.


\noindentА оптимальная политика действий:
$$
\phi* = \argmin_{\phi} \mathbb{E}_{\mathbf{S} \sim \mu_{0}} V_{\mathcal{M}}^{\phi}(\mathbf{s})
$$

    %!TEX root = ../data_poisoning.tex
\section{Атаки внедрением триггера (Backdoor)}
\subsection*{Цель}
Изменить набор обучающий данных $D$ таким образом, чтобы любое изображение, при наложении на него специального триггера классифицировалось в целевой класс $C'$.
При этом, также как и в случае обычных атак отравлением, нужно оказать как можно меньшее влияние на классификацию остальных изображений.

Примером триггера может быть помещение небольшого квадрата из чёрных или цветных пикселей в угол изображения.

\subsection*{Описание атак}

\textbf{Простая замена метки класса у отравленных изображений}
Также как и в случае обычных атак отравлением данных этот способ работает малоэффективно по тем же причинам.

\textbf{Clean-label backdoor}
Замена метки плоха ещё и тем, что посмотрев глазами на изображение можно легко убедиться, что на нём изображено не совпадает со значением метки класса.

Атака Clean-label backdoor заключается в том, чтобы уменьшить значимость всех иных признаков и заставить модель обращать максимальное внимание на триггер. Триггер при этом накладывается только на изображения класса $C'$, и таким образом метки классов остаются без изменения.

Для достижения поставленной цели могут быть использованы генеративные состязательные нейронные сети.
Пусть $G(z)$ – генератор, где $z$ – латентное представление изображения размерности $d$.

\noindentТогда найти латентное представление изображения $x$ можно по формуле:
$$E_{G}(x)=\arg \min _{z \in \mathbb{R}^{d}}\|x-G(z)\|_{2}$$

\noindentМы также можем сделать сдвиг в пространстве латентных представлений между двумя изображениям:
$$I_{G}\left(x_{1}, x_{2}, \tau\right)=G\left(\tau z_{1}+(1-\tau) z_{2}\right)$$

Где $\tau$ должно быть достаточно большим, чтобы снизить значимость признаков изображения, но достаточно маленьким, чтобы не быть заметным человеческому взгляду в отсутствии оригинала для сравнения.

% TODO: описать Adversarial examples bounded in `p-norm

\textbf{Hidden-trigger backdoor}
% TODO: описать Hidden-trigger backdoor

% \textbf{Deep Feature Space Trojan}
% Идея заключается в том чтобы сделать триггер более незаметным: для стороннего наблюдателя картинки с триггером и без него должны быть идентичны.
% Также триггер не должны обнаруживать системы, которые полагаются на
% то, что отравленные модели переобучаются на простые триггеры.
% TODO: разобраться можно ли применить без контроля обучения модели
    %!TEX root = ../data_poisoning.tex
\section{Методы защиты}
\subsection*{Поиск отравленных примеров при помощи кластеризации}
Идея метода детектирования: провести тренировочный набор данных через
обученную нейронную сеть и набор векторов активаций последнего скрытого
слоя. Этот набор разбивается по классам и в каждом классе отдельно хорошо
разделяется на кластеры соответствующие отравленным и чистым данным.
При этом перед кластеризацией используется PCA с проекцией на 3 первых
компоненты.

Самым эффективным методом кластеризации выбран k-means с $k=2$ Далее для проверки, что данные отравлены мы можем исключить какой-нибудь
кластер из обучающего множества и затем провести его через новый классификатор. Если большая доля примеров попадает в один конкретный неверный
класс, то это подозрительно и требует проверки.
Также когда отравленных данных $p\%$, то размер кластера, соответствующего им будет составлять примерно $p\%$. Если же отравленных данных нет,
то размеры кластеров будут примерно равны. С той же целью можно использовать silhouette score.
Также мы можем составить усреднённый пример из кластера и проверив его вручную сделать вывод о наличии отравления: например триггеры атак внедрения backdoor могут сильно выделяться.
Для исправления модели предлагается изменить метки отравленным
классам и дообучить модель на исправленных данных.

\subsection*{Поиск отравленных примеров при помощи Auto Encoder моделей}
Модель Auto Encoder можно использовать для детектирования сильных аномалий, потому что он не может их точно восстановить. Но для этого Auto Encoder должен Быть обучен на доверенном наборе данных.

Однако в чистом виде этот метод обладает несколькими недостатками:
\begin{itemize}
    \item Отравление часто происходит малыми изменениями, что может быть незаметным для Auto Encoder модели;
    \item Такой подход не анализирует метки классов, а они могут быть изменены;
    \item Требование доверенного набора данных накладывает существенные ограничения на применимость этого метода.
\end{itemize}

Для решения этих проблем был разработан метод под названием
"Classification Auto-Encoder based detector $+$".

%TODO: описать метод подробнее

\subsection*{Сертификация робастности модели}

Для некоторых моделей возможно доказать их устойчивость к отравлениям. Например для Bagging классификатора.
Обучающие подмножества для каждого базового классификатора выбираются случайно. Процесс обучения тоже носит стохастический характер.

Теоретическое определение вероятности того, что предсказанная метка при отравлении не изменится, получается при помощи леммы Неймана-Пирсона.
Определяется нижняя граница вероятности наиболее вероятного
предсказания и верхняя граница вероятности второго по вероятности предсказания для того чтобы при отравлении $r$ примеров не произошло изменений в предсказании.

$r$ получается как решение оптимизационной задачи.

%TODO описать задачу


\subsection*{Метод очистки отравленной модели GangSweep}
Ключевым понятием этого метода является "изменяющая маска" – минимальное попиксельное изменение изображения, при котором модель начинает классифицировать изображение из класса $C$ как изображение из класса $C'$.

По результатам экспериментов, внедрение backdoor слабо изменяет дисперсию в пространстве признаков, но сильно изменяет расстояние.

Состязательные нейросети при генерации изменяющей маски очень хорошо перебирают
пространство вокруг объекта. В методе используется состязательная сеть
для генерации изменяющей маски от каждого класса к каждому.
Если в модель внедрён Backdoor, то для различных изображений изменяющие маски к целевому классу будут очень близки.

Отфильтровать изменяющие маски для последующего анализа можно посредством алгоритма z-score, для того чтобы выделить те, которые слабо изменяют дисперсию, но сильно изменяют расстояние в пространстве признаков.

% TODO: добавить изображения

    %!TEX root = ../data_poisoning.tex
\section{Практические результаты}

Тут будет таблица с результатами экспериментов
    %!TEX root = ../data_poisoning.tex
\phantomsection
\section*{Вывод}
\addcontentsline{toc}{section}{Вывод}
Обзор современных методов атак отравлением обучающих данных на нейросетевые модели показывает, что на данный момент такие атаки активно развиваются и представляют собой реальную угрозу ввиду своей эффективности в определенных конфигурациях. Например, при определённых условиях, при отравлении 1\% обучающего набора данных можно достигнуть поставленной цели на более чем 90\% моделей обученных на этих данных.~[3]


Тут выводы о практических результатах.
}


% Информация о годе выполнения работы
\def\Year{%
    % 2006%
    \the\year%     % Текущий год
}

% Укажите тип работы
% Например:
%     Выпускная квалификационная работа,
%     Магистерская диссертация,
%     Курсовая работа, реферат и т.п.
\def\WorkType{%
    % Выпускная квалификационная работа%
    % Магистерская диссертация%
    Курсовая работа%
    % Реферат%
    % Дипломная работа%
}

% Название работы
%%%%%%%%%%% ВНИМАНИЕ! %%%%%%%%%%%%%%%%
% В МГУ ОНО ДОЛЖНО В ТОЧНОСТИ
% СООТВЕТСТВОВАТЬ ВЫПИСКЕ ИЗ ПРИКАЗА
% УТОЧНИТЕ НАЗВАНИЕ В УЧЕБНОЙ ЧАСТИ
\def\Title{%
    Исследование атак отравлением данных и методов защиты от них%
}


% Имя автора работы
\def\Author{%
    Лозинский Иван Павлович%
}

% Информация о научном руководителе
%% Фамилия Имя Отчество%
\def\SciAdvisor{%
    Саада Даниель Фирасович%
}
%% В формате: И.~О.~Фамилия%
\def\SciAdvisorShort{%
    Д.~Ф.~Саада%
}
%% должность научного руководителя
\def\Position{%
    % профессор%
    % доцент%
    % старший преподаватель%
    % преподаватель%
    % ассистент%
    % ведущий научный сотрудник%
    % старший научный сотрудник%
    % научный сотрудник%
    младший научный сотрудник%
}
%% учёная степень научного руководителя
\def\AcademicDegree{%
    % д.ф.-м.н.%
    % д.т.н.%
    % к.ф.-м.н.%
    % к.т.н.%
    без степени%
}

% Информация об организации, в которой выполнена работа
%% Город
\def\Place{%
    Москва%
}
%% Университет
\def\Univer{%
    Московский государственный университет имени М.~В.~Ломоносова%
}
%% Факультет
\def\Faculty{%
    Факультет вычислительной математики и кибернетики%
}
%% Кафедра
\def\Department{%
    Кафедра информационной безопасности%
}

%%%% Переключите статус документа для отладки
%%%% В режиме draft документ собирается очень быстро
%%%% и выводится полезная информация о том
%%%% какие строки вылезают за границы документа, что удобно для борьбы с ними
\def\Status{%
    % draft%
    final%
}

%%%% Включает и выключает подпись <<С текстом работы ознакомлен>>
\def\EnableSign{%
    % true%
}
